\documentclass[a4paper,11pt]{article}

\usepackage{cmap}		
%\usepackage[utf8]{inputenc}			
\usepackage[spanish]{babel}
\usepackage{framed}
\usepackage{hyperref}
\usepackage{amsmath}
\usepackage{graphicx}
\usepackage[colorinlistoftodos]{todonotes}
\usepackage{wrapfig}
\usepackage{lipsum}
\usepackage{listings}
\usepackage{color}
\usepackage{indentfirst}
\usepackage{times}
\usepackage{textcomp}
\usepackage{pgfgantt}
\usepackage{lipsum}
\usepackage{csquotes}

% set document font, font sizes, margin dimensions and spacing
\usepackage{fontspec}
\setmainfont{Arial}
\usepackage[top=15mm,bottom=25mm,left=20mm,right=20mm]{geometry}
\usepackage{setspace}\onehalfspacing
\usepackage{titlesec}
\titleformat*{\section}{\Large\bfseries}
\titleformat*{\subsection}{\Large\bfseries}
\titleformat*{\subsubsection}{\Large\bfseries}
\titleformat*{\paragraph}{\Large\bfseries}
\titleformat*{\subparagraph}{\Large\bfseries}
\setlength{\parskip}{0.6em}

\newif\ifblackandwhite
\blackandwhitetrue

\usepackage{etoolbox}
\usepackage{longtable}%
\AtBeginEnvironment{longtable}{%
  \addfontfeature{RawFeature=+tnum;-onum}%  <--- requires LuaTeX
}

\usepackage{pdflscape}
%\usepackage[svgnames]{xcolor}
 \usepackage{colortbl}%
   \newcommand{\myrowcolour}{\rowcolor[gray]{0.925}}
\usepackage{booktabs}

\ifblackandwhite
  \newcommand{\cheading}[2]{\textbf{#1\hfill #2}}
  \newcommand{\highest}[1]{\textbf{#1}}% == highest score for question
\else
  \newcommand{\cheading}[2]{\textcolor{Maroon}{\textbf{#1\hfill #2}}}
  \newcommand{\highest}[1]{\textcolor{Maroon}{\textbf{#1}}}%
\fi

\definecolor{mygray}{rgb}{0.4,0.4,0.4}
\definecolor{mygreen}{rgb}{0,0.8,0.6}
\definecolor{myorange}{rgb}{1.0,0.4,0}

\lstdefinestyle{customc}{
  belowcaptionskip=1\baselineskip,
  breaklines=true,
  frame=L,
  xleftmargin=\parindent,
  language=C,
  showstringspaces=false,
  basicstyle=\footnotesize\ttfamily,
  keywordstyle=\bfseries\color{green!40!black},
  commentstyle=\itshape\color{purple!40!black},
  identifierstyle=\color{blue},
  stringstyle=\color{orange},
  numbers=left,
  numbersep=12pt,
  numberstyle=\small\color{mygray},
}
\lstset{escapechar=@,style=customc}

\newcommand{\HRule}{\rule{\linewidth}{0.5mm}}

%Definindo um comando todoin que aceita quebra de linha e fórmulas
\newcommand\todoin[2][]{\todo[inline, caption={2do}, #1]{
\begin{minipage}{\textwidth-4pt}#2\end{minipage}}}

\newcommand\todogeg[2][]{\todo[inline, caption={#2}, color=yellow!100, #1]{
\begin{minipage}{\textwidth-4pt}#2\end{minipage}}}

\newcommand\todovwcm[2][]{\todo[inline, caption={#2}, color=red!100, #1]{
\begin{minipage}{\textwidth-4pt}#2\end{minipage}}}
\begin{document}
\begin{titlepage}
\begin{center}

% logo
\includegraphics[width=0.5\textwidth]{images/logo-udea.png}~
\\[2cm]



% identificação do relatório
\HRule \\[0.4cm]
{\large \bfseries NACIMIENTO DE LA COMPUTACIÓN: DE LA CRISIS A LA REVOLUCION \\
[0.4cm]}
\HRule 
\\[2cm]

% identificação do aluno
\large\textbf{ALUMNO}\\[1cm]
ANDREA OSPINA HINCAPIÉ\\
andrea.ospinah@udea.edu.co\\
INFORMÁTICA II \\
 INGENIERÍA ELECTRÓNICA
\\[1.5cm]

% identificação do orientador
\large\textbf{DOCENTE}\\[1cm]
AUGUSTO ENRIQUE SALAZAR JIMÉNEZ\\
DEPARTAMENTO DE INGENIERÍA ELECTRÓNICA Y TELECOMUNICACIONES
\\[1cm]



\vfill

% Bottom of the page
{\large \textbf{27 DE MARZO DE 2020}}

\end{center}
\end{titlepage}

\newpage
\centerline{{\large\bfseries NACIMIENTO DE LA COMPUTACIÓN: DE LA CRISIS A LA REVOLUCIÓN}}
Es el año 2014 y en salas de cine acaba de estrenarse una nueva película ambientada en la Segunda Guerra Mundial, y que dice estar basada en hechos reales, titulada The Imitation Game \cite{imdb_2014}.Durante sus dos horas de duración, la película se encargará de presentar una historia que, independiente de si es una representación fiel de los hechos históricos, logrará cautivar a miles de espectadores en el aniversario de los sesenta años de la muerte de su protagonista: Alan Turing; un hombre que, como narra la película, contribuyó no sólo a la derrota de Adolf Hitler, sino también al nacimiento de la computación moderna.\\
Sin embargo, es probable que los espectadores terminen de ver esta historia y utilicen palabras como “trágico” o “amargo” para describir su desenlace. Después de todo, por ser gay, Turing fue acusado en 1952 por el crimen de “gross indecency”, que puede traducirse como una “conducta indecente” o una “conducta inmoral” y cuya condena, la castración química, se afirma que lo llevó al suicidio en 1954 \cite{grimstad_2020}.\\
Y así como sucede con palabras como “trágico” que se relacionan estrechamente con los lamentables hechos de la vida de Turing, podría decirse que sus ideas ligadas con el nacimiento de la computación tienen una estrecha relación con otra palabra que suele tener una connotación negativa: “crisis”.\\
Pero, ¿a qué “crisis” podrían estar ligadas? No se tratan de crisis económicas o crisis generadas por una pandemia mundial, sino a la Crisis de Fundamentos de la matemática, un hecho que durante el siglo XX \cite{chaitin2003ordenadores} fue clave pues gracias a mentes como las de Georg Cantor, David Hilbert, Kurt Gödel y, por supuesto, Alan Turing, se determinaron los límites del razonamiento matemático y se introdujeron ideas, como las de Turing, a partir de las cuales surgió la computación moderna.\\
Es por esto que, y tal como en el análisis de importantes revoluciones a lo largo de la historia siempre deben ser incluidos los turbulentos antecedentes, es necesario introducir la Crisis de Fundamentos para comprender el nacimiento de una revolución como la fue el nacimiento de la computación moderna: \\ 
{\large\bfseries LA CRISIS:} \\
Ahora es el año 1845 y en San Petersburgo, Rusia \cite{dusautoy_2018}
ha nacido un hombre fundamental en la historia de la matemática: Georg Cantor. El mundo en el que ha nacido, construido en gran medida sobre bases matemáticas y científicas, ha sido influenciado por las ideas aristotélicas del infinito y su “concepción potencial”\cite{crespo2006paseo}, la cual está relacionada con la forma más intuitiva del infinito: “(…) Se lo interpreta como aquello que siendo de hecho finito, crece y puede crecer sin límite alguno” \cite[p.2]{crespo2006paseo}. Sin embargo, con ésta concepción surgen paradojas como la de Zenón, los enunciados de Proclo de Alejandría y Galileo Galilei, de las que se obtienen conclusiones imposibles de aceptar en la matemática del momento que: "[da] al infinito un tratamiento similar al que se da a conjuntos finitos" \cite[p.2]{crespo2006paseo}.Así, el infinito estaba permitiendo vislumbrar una crisis de fundamentos. Sin embargo, todo estaría a punto de cambiar con Georg Cantor y sus estudios del infinito desde el punto de vista de la coordinabilidad de conjuntos y los números transfinitos al definir al infinito como: “(…) aquel en el que se puede establecer una correspondencia uno a uno entre el mismo conjunto y una parte propia de él” \cite[p.5]{crespo2006paseo}. \\
Sus ideas no fueron bien recibidas inicialmente por algunos destacados matemáticos. Leopold Krönecker, un matemático alemán, se opuso a estas hasta el punto calificar a Cantor como un “corruptor de la juventud” \cite{crespo2006paseo}, por lo que fue un claro ejemplo de la actitud de la comunidad matemática frente a Cantor, el cual tuvo dificultades para publicar sus ideas; hecho que empeoró su salud mental y lo llevaría a pasar sus últimos años en un hospital psiquiátrico \cite{dusautoy_2018}. \\
Pero de las ideas de Cantor no todos fueron detractores: Dedekind se encargó de defender el enfoque conjuntista de Cantor, el cual: “(…) estuvo acompañado por nuevas concepciones de los fundamentos de la matemática que estimularon la aparición de sistematizaciones” \cite[p.5]{crespo2006paseo} y otro personaje crucial en la historia de la Crisis de los Fundamentos, David Hilbert, utilizó las ideas de Cantor en su propio trabajo e incluso llegó a afirmar que: “Del paraíso que Cantor nos
creó, nadie podrá expulsarnos.” \cite[p.5]{crespo2006paseo}
La historia de David Hilbert es, entonces, la siguiente pieza de esta crisis. Si se avanza en el tiempo a la Alemania de 1862 \cite{barral_2017} será posible encontrarse en la tierra natal y en el año de nacimiento de este influyente matemático y activista, el cual durante sus años como docente en la Universidad de Gotinga manifestó su postura contra la discriminación de los profesores judíos durante la Alemania Nazi, la participación de Alemania en la guerra e, incluso, intentó que la Universidad de Gotinga aceptara a Emmy Noether como profesora, tras ser rechazada por ser mujer \cite{barral_2017}. Pero, ¿fue exitoso? La respuesta podría ser que no: los profesores judíos fueron expulsados, se presentó la Segunda Guerra Mundial y Emmy Noether no fue aceptada como profesora, una serie de fracasos a la que podría sumársele también el “fracaso” que le abriría la puerta a la computación moderna, a pesar de que la palabra “fracaso” para referirse a los estudios matemáticos de Hilbert solo cuenta una historia a medias, pues sobre éstos se basan importantes ramas de la matemática como el análisis funcional o la matemática física \cite{barral_2017}.\\
Para hablar de este “fracaso” es entonces necesario conocer el plan de Hilbert, el cual tiene su origen tanto en la crisis que habían creado las ideas de Cantor en la matemática del momento, como en la crisis de la lógica que se presentaba por paradojas como la paradoja de Russell o la paradoja del mentiroso \cite{chaitin2003ordenadores}; aquella en la que si alguien afirma que:” ¡lo que estoy diciendo es falso!”, tras un momento de análisis sólo se podría afirmar que no se sabe si este mentiroso es en realidad un mentiroso. \\
Son estas crisis las que Hilbert plantea resolver cuando, por medio del formalismo, propone: “(…) crear para el razonamiento, para la deducción y para la matemática un lenguaje artificial perfecto” \cite[p.3]{chaitin2003ordenadores}, estando este compuesto por axiomas definidos de una manera rigurosa y a partir de los cuales podría demostrarse y obtener todos los teoremas de la matemática \cite{chaitin2003ordenadores}, por lo que se encontraría solución a todos los problemas que se planteasen \cite{barral_2017}; un sistema que mostraría el inmenso potencial de las matemáticas y que podría resumirse con la frase que se encuentra en su epitafio: “Debemos saber. ¡Sabremos!” \cite{bombal_2018}. \\
Pero pronto el plan de Hilbert fracasaría. Es ahora el año 1930 y Kurt Gödel, un matemático nacido en la actual República Checa \cite{rago_2016}, ha publicado su “Teorema de la incompletitud” con el cual derrumbará el plan de Hilbert al demostrar que un sistema de axiomas no puede ser completo y consistente al mismo tiempo \cite{rago_2016}, por lo que: “(…) si se supone que los axiomas y las reglas de deducción no permiten la demostración de teoremas falsos, habrá teoremas verdaderos que no podrán ser demostrados” \cite[p.4]{chaitin2003ordenadores}. \\
Este hecho, naturalmente, sacude el mundo matemático y algunos de sus miembros, como John von Neumann, inmediatamente aceptan y admiran el trabajo de Gödel \cite{chaitin2003ordenadores}, mientras que otros como Alan Turing, que trabajaban de manera paralela, continúarán estudiando los límites de la matemática a través de los problemas presentados en el plan de Hilbert y, es en este punto que, gracias a las discusiones planteadas en la Crisis de los Fundamentos, surge la computación moderna. 
El trabajo de Gödel además, es la base de numerosas áreas de estudio de la lógica matemática como la teoría de la recursión y la teoría de los modelos \cite{chaitin2003ordenadores}, y a lo largo de su vida fue docente en prestigiosas universidades como la Universidad de Viena y la Universidad de Princeton, a la cual llegaría tras abandonar Viena al ser Austria  anexado a la Alemania Nazi \cite{rago_2016}.Desafortunadamente, como Cantor, Gödel sufrió de diversas enfermedades mentales en los últimos años de su vida, una de las cuales, la manía de persecución, lo llevaría a la muerte al negarse a consumir alimentos no preparados por su esposa al estar convencido de que sus compañeros de Princeton querían envenenarlo \cite{baro_pizarro_2019}.\\  
{\large\bfseries LA REVOLUCIÓN:}\\
Es así como ahora se llega al año 1936 \cite{chaitin2003ordenadores}, en el que Alan Turing publicará un artículo en el que contestará el “problema de decisión” propuesto por Hilbert:
\begin{displayquote}
 Dados ciertos principios básicos de una teoría y una posible consecuencia de estos principios, ¿existe un ‘procedimiento efectivo’ para saber si el candidato a teorema es una consecuencia lógica de los axiomas, y por lo tanto amplía nuestro conocimiento de la teoría? \cite[p.5]{hernández_2013}
\end{displayquote}
Para dar su respuesta a esta pregunta, Turing se enfoca entonces en describir lo que es un “procedimiento efectivo” y es, a partir de esta descripción, que se construye la computación moderna. 
Inspirado en el trabajo de las personas cuyo trabajo era realizar cálculos manuales, Turing describió el “proceso efectivo” como un proceso que: “(…) estaba basado en un conjunto finito de instrucciones que se reducían, en última instancia, a la manipulación de símbolos” \cite[p.6]{hernández_2013} y de esta manera, por medio de la abstracción, crea la “máquina de Turing”, la cual contaría con elementos como: un conjunto de símbolos, estados de máquina (inicio, final, etc.), un conjunto de instrucciones, una larga tira de papel que contiene símbolos y un lector \cite{hernández_2013}, \cite{chaitin2003ordenadores}; de manera que la máquina pudiese cumplir una función determinada, y que un conjunto de instrucciones y procedimientos que componían a máquinas particulares, y que fueran leídos por una sola máquina, formasen una “máquina universal”\cite{hernández_2013}. ¿Suena familiar? Pues es porque sobre esta descripción teórica se construyeron los computadores que se tienen en la actualidad.
La “máquina de Turing” le permite entonces definir lo que es o no un procedimiento computable según si se puede o no resolver utilizando estas máquinas \cite{hernández_2013}. Además, Turing estudia los límites de la computación a través del “problema de la detención”, el cual se pregunta sobre si es posible construir una máquina que determine si el proceso realizado por otra terminará o no. ¿su respuesta? Esto no es posible pues se obtendrán contradicciones similares a las obtenidas con la paradoja del mentiroso y, por este motivo el “procedimiento efectivo” propuesto en el “problema de decisión” no será posible para todos los teoremas \cite{chaitin2003ordenadores}, \cite{hernández_2013}.
Y es así como a través de una “crisis” se obtienen importantes avances que tienen su eco en el desarrollo tecnológico del último siglo. Por supuesto, a lo largo de la historia de la computación y en su origen se han introducido otros modelos diferentes al de Turing como el “cálculo lambda” de Alonzo Church (1903-1995) o la teoría de “funciones recursivas” de Gödel (que eventualmente se probaría que son equivalentes), cuyo estudio y el de otros importantes personajes para comprender por completo el origen de la computación y de lenguajes como Lisp o ALGOL \cite{hernández_2013}, \cite{baez2010physics} se sale del ámbito de la computación estudiado por medio de este ensayo.
Es así como se puede ahora avanzar en el tiempo al año actual, el 2020, en el que existe una sociedad construida sobre una larga serie de “crisis” seguidas por “revoluciones” que parecen no tener un final definido, pues en áreas como la computación es posible atisbar el futuro. Es posible entonces preguntarse, como los matemáticos que se preguntaron por los límites del poder de la matemática, ¿cuál es el límite del desarrollo de la computación a la que es posible acercarse? La pregunta sigue abierta pero como se puede observar en la historia de la computación: de las pocas respuestas surgen los mayores avances.  

\nocite{maestre_timón_2018}



\newpage

\renewcommand\refname{REFERENCIAS BIBLIOGRÁFICAS}
%\bibliographystyle{abntex2-alf}
\bibliographystyle{IEEEtran}
\bibliography{referencias.bib}
%\listoftodos

\end{document}